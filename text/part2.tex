%!TEX root = ../zeeman.tex
С математической точки зрения квантовые числа \textbf{\textsl{ L,S,J}} определяют собственные значения уравнения Шредингера $\mathbf{\mathit{P_L,P_S,P_J}}$ 
для \textbf{волновой функции} электронов $\Psi$ в центрально - симметричном поле ядра, а дискретность собственых значений связана с тем, что волновая функция должна однозначно описывать состояние электронного облака в данной точке, то есть быть периодической функцией угла:

\begin{equation}
\Psi(\Theta+2\pi)=\Psi(\Theta)
\label{eq:4} 
\end{equation}

Условие (\ref{eq:4}) приводит к следующемму \textbf{правилу квантования} проекции момента импульса на фиксированное направление (например, на направление $\vec{H}$):

\begin{equation}
P_{J_H} = \hbar M_J
\label{eq:5} 
\end{equation}

где квантовое число $M_{J}$ принимает $2J+1$ значений: $J,J-1,..0,...-J$.  При этом модуль момента импульса равен: 

\begin{equation}
P_{J}=\hbar \sqrt {J(J+1)}
\label{eq:6} 
\end{equation}

Квантовое число $J$ в силу различной возможной ориентации векторов $\vec P_S$ (см. формулу (\ref{eq:1})) принимает следующие значения: $$J=L+S,L+S-1,...,|L-S|.$$

Для расчета зеемановского расщепления по формулам (\ref{eq:2}),(\ref{eq:3}) необходимо связать вычисленный по формуле (\ref{eq:6}) полный механический момент атома со средней проекцией 
$\mu_H$ его магнитного момента на направление $\vec H$. Вид этой связи зависит от квантового состояния атома, а ее вычисление представляет собой отдельную задачу, решение которой будет дано в п.3 в достаточно простом приближении. Пока же для получения общей формулы зеемановского расщепления представим связь $\mu_H$ и $P_J$ феноменологически в виде: 

\begin{equation}
\mu_{H}=\mu_{0} g \frac{P_{J_H}}{\hbar}=\mu_{0} g M_{J},
\label{eq:7} 
\end{equation}

где \textbf{g} - так называемый \textbf{фактор} (или \textbf{множитель}) \textbf{Ланде} соответствующего квантового состояния, а величина $\mu_0$ называется \textbf{магнетоном Бора} и имеет смысл наименьшей отличной от нуля проекции магнитного момента, связанного с орбитальным движением электрона в атоме: 
$\displaystyle{\mu_{0} = {\frac{e\hbar}{2mc}}=9.27\cdot10^{-21}}$
эрг/Гс.
Здесь и далее: e - элементарный заряд, m - масса электрона, с - скорость света.

Будем считать, что внешнее магнитное поле достаточно слабо, так что $\mu_0 H$ много меньше разности энергий между любой парой  рассматриваемых уровней атома. Тогда зеемановское расщепление интересующего нас уровня $E$ можно рассматривать изолированно и в соответствии с (\ref{eq:2}), (\ref{eq:5}) и (\ref{eq:7}) написать: 
\begin{equation}
\mathbf{\delta E_{M} = - g \mu_{0} H M_J}.
\label{eq:8} 
\end{equation}  

Таким образом, при переходе между каждой парой подуровней, образовавшихся в результате расщепления в магнитном поле энергетических состояний, описываемых квантовыми числами $(n_{2},J_{2},L_{2},S_{2})$ и $(n_{1},J_{1},L_{1},S_{1})$  будут излучать частоты: 
\begin{equation}
\omega_{1,2} = \frac{E_{2}(n_{2},J_{2},L_{2},S_{2},M_{J_{2}})-E_{1}(n_{1},J_{1},L_{1},S_{1},M_{J_1})}{\hbar}
\label{eq:9} 
\end{equation}


Однако не все из указанных переходов могут быть осуществлены. Действительно, так как квант электромагнитного излучения (фотон) имеет отличный от нуля собственный момент импульса (спин) $P_{\mu_H} = {0,\pm\hbar}$, то из закона сохранения момента импульса и формулы (\ref{eq:5}) следует, что в процессе излучения магнитное квантовое числоа $M_J$ атома  может либо измениться на удиницу ($\Delta M=\pm 1$),  либо остаться еизменным ($\Delta M = 0$) . Можно показать, что аналогичные ограничения накладываются на изменение квантового числа $J$ : $\Delta J = {0,\pm 1}$.

Указанные усдлвия носят название правил отбора и определяют допустимые переходы медлу щеемановскими уровнями. Соответствующие им зеемановские линии в спектре излучения носят название $\mathbf{\pi}$ - (при $\Delta M = 1$) и $\mathbf{\sigma }$ - (при $\Delta M= \pm 1$)  компонент и отличаются, в частности, поляризацией. Вдоль магнитного поля излучаются лишь циркулярно поляиризованные $\sigma$ - компоненты,  поляризованные линейно в перпендикулярных друг другу плоскостях. В зависимости от направления наблюдения говорят соответственно о \textbf{продольном} и \textbf{поперечном} \textit{эффекте Зеемана}. 

Приведенные выше формулы позволяют легко рассчитать вид зеемановского спектра. Пусть в отсутствие внешнего магнитного поля при переходе между уровнями $E_{1}(J_{1},L_{1},S_{1})$ и $E_{2}(J_{2},L_{2},S_{2})$, которые далее будем называть \textbf{комбинирующими}, излучается линия с частотой $\omega_0$. При наложении поля расщепления каждого из комбинирующих уровней будет определяться формулой (\ref{eq:8}), и в соответствии с правилами отбора в системе станут возможны переходы между уровнями с квантовыми числами: 

$$(J_{1},L_{1},S_{1},M)\rightarrow(J_{2},L_{2},S_{2},M)$$
$$(J_{1},L_{1},S_{1},M \pm1)\rightarrow(J_{2},L_{2},S_{2},M)$$
с излучением частот $\omega_{M_1,M_2}$
$$\omega_{M2,M1}=\frac{E_{M2}-E_{M1}}{\hbar}$$
$$\omega_0=\frac{E_{02}-E_{01}}{\hbar}$$
$$E_1=E_{01}-(\vec{\mu},\vec{H})=E_{01}-M_{J1}\mu_0g_1H$$
$$E_2=E_{02}-(\vec{\mu},\vec{H})=E_{02}-M_{J2}\mu_0g_2H$$


\begin{equation} \tag{10a}
	\omega_{M2,M1}=\frac{E_{02}-M_{J2}\mu_0g_2H-E_{01}+M_{J1}\mu_0g_1H}{\hbar}=\omega_0+\frac{h\mu_0}{\hbar}(M_{J1}g_1-M_{J2}g_2)
	\label{eq:10a} 
\end{equation}
\begin{equation} \tag{10b}
	\omega_{M_{\pm1},M}=\omega_0 + \frac{\delta E_{1,M\pm1} - \delta E_{2,M}}{\hbar} = \omega_0 \pm g_1 \frac {\mu_0 H}{\hbar} +(g_1 - g_2)M\frac{\mu_0 H}{\hbar} 
	\label{eq:10b} 
\end{equation}
смещенных относительно основной частоты $\omega_0$ на величину:
	\begin{equation}
	\addtocounter{equation}{1}
	\Delta \omega_{M_1,M_2} = \omega_{M_1,M_2} - \omega_{0} = (g_{1} M_{1} - g_{2} M_{2})\frac{\mu_{0} H}{\hbar}
	\label{eq:11} 
\end{equation}

Как видно из (\ref{eq:10a}), для различных по знаку M квантовых состояний $\pi$ - компоненты излучения расположены симметрично относительно несмещенной линии $\omega_0$, а $\sigma$ - компоненты каждой из двух поляризаций - симметрично относительно смещенных положений $\omega_0 \pm g_{1} \frac{\mu_{0} H}{\hbar}$. Расстояние между составляющими зеемановского спектра внутри каждой из трех групп пропорционально разности \textbf{g} - факторов комбинирующих уровней. К вычислению \textbf{g} - факторов квантовых состояний для достаточно простой модели атома мы и переходим ниже.



% \end{document}